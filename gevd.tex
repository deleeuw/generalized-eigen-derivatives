% Options for packages loaded elsewhere
\PassOptionsToPackage{unicode}{hyperref}
\PassOptionsToPackage{hyphens}{url}
\PassOptionsToPackage{dvipsnames,svgnames,x11names}{xcolor}
%
\documentclass[
  12pt,
  letterpaper,
  DIV=11,
  numbers=noendperiod]{scrartcl}

\usepackage{amsmath,amssymb}
\usepackage{iftex}
\ifPDFTeX
  \usepackage[T1]{fontenc}
  \usepackage[utf8]{inputenc}
  \usepackage{textcomp} % provide euro and other symbols
\else % if luatex or xetex
  \usepackage{unicode-math}
  \defaultfontfeatures{Scale=MatchLowercase}
  \defaultfontfeatures[\rmfamily]{Ligatures=TeX,Scale=1}
\fi
\usepackage{lmodern}
\ifPDFTeX\else  
    % xetex/luatex font selection
    \setmainfont[]{Times New Roman}
\fi
% Use upquote if available, for straight quotes in verbatim environments
\IfFileExists{upquote.sty}{\usepackage{upquote}}{}
\IfFileExists{microtype.sty}{% use microtype if available
  \usepackage[]{microtype}
  \UseMicrotypeSet[protrusion]{basicmath} % disable protrusion for tt fonts
}{}
\makeatletter
\@ifundefined{KOMAClassName}{% if non-KOMA class
  \IfFileExists{parskip.sty}{%
    \usepackage{parskip}
  }{% else
    \setlength{\parindent}{0pt}
    \setlength{\parskip}{6pt plus 2pt minus 1pt}}
}{% if KOMA class
  \KOMAoptions{parskip=half}}
\makeatother
\usepackage{xcolor}
\setlength{\emergencystretch}{3em} % prevent overfull lines
\setcounter{secnumdepth}{5}
% Make \paragraph and \subparagraph free-standing
\makeatletter
\ifx\paragraph\undefined\else
  \let\oldparagraph\paragraph
  \renewcommand{\paragraph}{
    \@ifstar
      \xxxParagraphStar
      \xxxParagraphNoStar
  }
  \newcommand{\xxxParagraphStar}[1]{\oldparagraph*{#1}\mbox{}}
  \newcommand{\xxxParagraphNoStar}[1]{\oldparagraph{#1}\mbox{}}
\fi
\ifx\subparagraph\undefined\else
  \let\oldsubparagraph\subparagraph
  \renewcommand{\subparagraph}{
    \@ifstar
      \xxxSubParagraphStar
      \xxxSubParagraphNoStar
  }
  \newcommand{\xxxSubParagraphStar}[1]{\oldsubparagraph*{#1}\mbox{}}
  \newcommand{\xxxSubParagraphNoStar}[1]{\oldsubparagraph{#1}\mbox{}}
\fi
\makeatother


\providecommand{\tightlist}{%
  \setlength{\itemsep}{0pt}\setlength{\parskip}{0pt}}\usepackage{longtable,booktabs,array}
\usepackage{calc} % for calculating minipage widths
% Correct order of tables after \paragraph or \subparagraph
\usepackage{etoolbox}
\makeatletter
\patchcmd\longtable{\par}{\if@noskipsec\mbox{}\fi\par}{}{}
\makeatother
% Allow footnotes in longtable head/foot
\IfFileExists{footnotehyper.sty}{\usepackage{footnotehyper}}{\usepackage{footnote}}
\makesavenoteenv{longtable}
\usepackage{graphicx}
\makeatletter
\newsavebox\pandoc@box
\newcommand*\pandocbounded[1]{% scales image to fit in text height/width
  \sbox\pandoc@box{#1}%
  \Gscale@div\@tempa{\textheight}{\dimexpr\ht\pandoc@box+\dp\pandoc@box\relax}%
  \Gscale@div\@tempb{\linewidth}{\wd\pandoc@box}%
  \ifdim\@tempb\p@<\@tempa\p@\let\@tempa\@tempb\fi% select the smaller of both
  \ifdim\@tempa\p@<\p@\scalebox{\@tempa}{\usebox\pandoc@box}%
  \else\usebox{\pandoc@box}%
  \fi%
}
% Set default figure placement to htbp
\def\fps@figure{htbp}
\makeatother
% definitions for citeproc citations
\NewDocumentCommand\citeproctext{}{}
\NewDocumentCommand\citeproc{mm}{%
  \begingroup\def\citeproctext{#2}\cite{#1}\endgroup}
\makeatletter
 % allow citations to break across lines
 \let\@cite@ofmt\@firstofone
 % avoid brackets around text for \cite:
 \def\@biblabel#1{}
 \def\@cite#1#2{{#1\if@tempswa , #2\fi}}
\makeatother
\newlength{\cslhangindent}
\setlength{\cslhangindent}{1.5em}
\newlength{\csllabelwidth}
\setlength{\csllabelwidth}{3em}
\newenvironment{CSLReferences}[2] % #1 hanging-indent, #2 entry-spacing
 {\begin{list}{}{%
  \setlength{\itemindent}{0pt}
  \setlength{\leftmargin}{0pt}
  \setlength{\parsep}{0pt}
  % turn on hanging indent if param 1 is 1
  \ifodd #1
   \setlength{\leftmargin}{\cslhangindent}
   \setlength{\itemindent}{-1\cslhangindent}
  \fi
  % set entry spacing
  \setlength{\itemsep}{#2\baselineskip}}}
 {\end{list}}
\usepackage{calc}
\newcommand{\CSLBlock}[1]{\hfill\break\parbox[t]{\linewidth}{\strut\ignorespaces#1\strut}}
\newcommand{\CSLLeftMargin}[1]{\parbox[t]{\csllabelwidth}{\strut#1\strut}}
\newcommand{\CSLRightInline}[1]{\parbox[t]{\linewidth - \csllabelwidth}{\strut#1\strut}}
\newcommand{\CSLIndent}[1]{\hspace{\cslhangindent}#1}

\usepackage{tcolorbox}
\usepackage{amssymb}
\usepackage{yfonts}
\usepackage{bm}


\newtcolorbox{greybox}{
  colback=white,
  colframe=blue,
  coltext=black,
  boxsep=5pt,
  arc=4pt}
  
\newcommand{\sectionbreak}{\clearpage}

 
\newcommand{\ds}[4]{\sum_{{#1}=1}^{#3}\sum_{{#2}=1}^{#4}}
\newcommand{\us}[3]{\mathop{\sum\sum}_{1\leq{#2}<{#1}\leq{#3}}}

\newcommand{\ol}[1]{\overline{#1}}
\newcommand{\ul}[1]{\underline{#1}}

\newcommand{\amin}[1]{\mathop{\text{argmin}}_{#1}}
\newcommand{\amax}[1]{\mathop{\text{argmax}}_{#1}}

\newcommand{\ci}{\perp\!\!\!\perp}

\newcommand{\mc}[1]{\mathcal{#1}}
\newcommand{\mb}[1]{\mathbb{#1}}
\newcommand{\mf}[1]{\mathfrak{#1}}

\newcommand{\eps}{\epsilon}
\newcommand{\lbd}{\lambda}
\newcommand{\alp}{\alpha}
\newcommand{\df}{=:}
\newcommand{\am}[1]{\mathop{\text{argmin}}_{#1}}
\newcommand{\ls}[2]{\mathop{\sum\sum}_{#1}^{#2}}
\newcommand{\ijs}{\mathop{\sum\sum}_{1\leq i<j\leq n}}
\newcommand{\jis}{\mathop{\sum\sum}_{1\leq j<i\leq n}}
\newcommand{\sij}{\sum_{i=1}^n\sum_{j=1}^n}
	
\KOMAoption{captions}{tableheading}
\makeatletter
\@ifpackageloaded{caption}{}{\usepackage{caption}}
\AtBeginDocument{%
\ifdefined\contentsname
  \renewcommand*\contentsname{Table of contents}
\else
  \newcommand\contentsname{Table of contents}
\fi
\ifdefined\listfigurename
  \renewcommand*\listfigurename{List of Figures}
\else
  \newcommand\listfigurename{List of Figures}
\fi
\ifdefined\listtablename
  \renewcommand*\listtablename{List of Tables}
\else
  \newcommand\listtablename{List of Tables}
\fi
\ifdefined\figurename
  \renewcommand*\figurename{Figure}
\else
  \newcommand\figurename{Figure}
\fi
\ifdefined\tablename
  \renewcommand*\tablename{Table}
\else
  \newcommand\tablename{Table}
\fi
}
\@ifpackageloaded{float}{}{\usepackage{float}}
\floatstyle{ruled}
\@ifundefined{c@chapter}{\newfloat{codelisting}{h}{lop}}{\newfloat{codelisting}{h}{lop}[chapter]}
\floatname{codelisting}{Listing}
\newcommand*\listoflistings{\listof{codelisting}{List of Listings}}
\makeatother
\makeatletter
\makeatother
\makeatletter
\@ifpackageloaded{caption}{}{\usepackage{caption}}
\@ifpackageloaded{subcaption}{}{\usepackage{subcaption}}
\makeatother

\usepackage{bookmark}

\IfFileExists{xurl.sty}{\usepackage{xurl}}{} % add URL line breaks if available
\urlstyle{same} % disable monospaced font for URLs
\hypersetup{
  pdftitle={Differentiating Generalized Eigen and Singular Value Decompositions},
  pdfauthor={Jan de Leeuw},
  colorlinks=true,
  linkcolor={blue},
  filecolor={Maroon},
  citecolor={Blue},
  urlcolor={Blue},
  pdfcreator={LaTeX via pandoc}}


\title{Differentiating Generalized Eigen and Singular Value
Decompositions}
\author{Jan de Leeuw}
\date{February 3, 2025}

\begin{document}
\maketitle
\begin{abstract}
The derivatives of eigenvalues and eigenvectors (and singular values and
vectors) are used in many places in multivariate data analysis. This
paper reviews a number of formulas for these derivatives and discusses
several applications. The results extend, generalize, correct, and
improve the results of De Leeuw (\citeproc{ref-deleeuw_R_07c}{2007}).
\end{abstract}

\renewcommand*\contentsname{Table of contents}
{
\hypersetup{linkcolor=}
\setcounter{tocdepth}{3}
\tableofcontents
}

\sectionbreak

\textbf{Note:} This is a working manuscript which will be
expanded/updated frequently. All suggestions for improvement are
welcome. All qmd, tex, html, pdf, R, and C files are in the public
domain. Attribution will be appreciated, but is not required.

\sectionbreak

\section{Introduction}\label{sec-intro}

Suppose \(A\) and \(B\) are real symmetric matrices of order \(n\), with
\(B\) positive definite (PSD, from now on). Generalized eigenvalues and
eigenvectors are defined as the solutions \((x,\lambda)\) of the system
of equations \begin{subequations}
\begin{align}
Ax&=\lambda Bx,\label{eq-evd1}\\
x'Bx&=1.\label{eq-evd2}
\end{align}
\end{subequations} We call this a GEV system, short for generalized
eigenvalue system.

The properties of the solutions of the system
\eqref{eq-evd1},\eqref{eq-evd2} can be found in any textbook on matrix
algebra, for example in my personal favorite Wilkinson
(\citeproc{ref-wilkinson_65}{1965}). We briefly summarise them here.

Solving equation \eqref{eq-evd1} is equivalent to solving the
determinantal equation \(\text{det}(A-\lambda B)=0\). The polynomial
\(\text{det}(A-\lambda B)\) is of degree \(n\), and consequently has
\(n\) roots. Because \(A\) is symmetric and \(B\) is positive definite
all \(n\) roots are real. Note that \(A\) can be indefinite and/or
singular, which means that some roots can be negative or zero.

If \(\lambda_s\not=\lambda_t\) are two solutions of the determinantal
equation then the corresponding eigenvectors, which are defined up to a
scale factor, \(x_s\) and \(x_t\) are B-orthogonal,
i.e.~\(\smash{x_s'Bx_t=0}\). If \(\lambda_s\) is a root of multiplicity
\(p\) then there are \(p\) corresponding eigenvectors, spanning a
\(p\)-dimensional subspace of \(\mathbb{R}^n\), and these \(p\)
eigenvectors can be chosen to be B-orthogonal as well. If we use the
normalization in \eqref{eq-evd2} it follows that there exists a
non-singular matrix \(X\) and a diagonal \(\Lambda\) such that
\(X'BX=I\) and \(AX=BX\Lambda\), which implies \(X'AX=\Lambda\). If all
roots are different the solution \((X,\Lambda)\) is unique up to a
permutation of the columns of \(X\), and we can eliminate this
non-uniqueness by requiring that
\(\lambda_1>\lambda_2>\cdots>\lambda_n\). If there are multiple roots,
and we require \(\lambda_1\geq\lambda_2\geq\cdots\geq\lambda_n\), the
solution is unique up to a rotation within each of the subspaces
associated with multiple roots.

If \(\lambda_s\) is a simple eigenvalue, i.e.~it is different from all
other roots, then both \(\lambda_s\) and \(x_s\) are differentiable at
\((A,B)\) (see Wilkinson (\citeproc{ref-wilkinson_65}{1965}), chapter 2,
and for much more detail Kato (\citeproc{ref-kato_76}{1976})). Suppose
the matrices \(\Delta_A\) and \(\Delta_B\) are real and symmetric
perturbations.
Define\footnote{Here $o(\epsilon)$ is any function of $\epsilon$ satisfying 
$\lim_{\epsilon\rightarrow 0}o(\epsilon)/\epsilon=0$.}\footnote{The symbol $:=$ is used for definitions.}
\begin{subequations}
\begin{align}
A(\epsilon)&:=A+\epsilon\Delta_A+o(\epsilon),\\
B(\epsilon)&:=B+\epsilon\Delta_B+o(\epsilon).
\end{align}
\end{subequations} Differentiability of \(\lambda_s\) and \(x_s\)
implies that the differentials \begin{subequations}
\begin{equation}
d\lambda_s:=\lim_{\epsilon\rightarrow 0}\frac{\lambda_s(A(\epsilon),B(\epsilon))-\lambda_s(A,B)}{\epsilon},\label{eq-gder1}
\end{equation}
and 
\begin{equation}
dx_s:=\lim_{\epsilon\rightarrow 0}\frac{x_s(A(\epsilon),B(\epsilon))-x_s(A,B)}{\epsilon}\label{eq-gder2}
\end{equation}
\end{subequations} exist. Our notation surpresses the dependence of
\(d\lambda_s\) and \(dx_s\) on \((A,B)\) and on \((\Delta_A,\Delta_B)\)
because for our purposes these are just fixed constants.

\sectionbreak

\section{Perturbation}\label{sec-perturb}

\subsection{Basic Perturbations}\label{sec-basic}

We study the effect of symmetric perturbations
\(A+\epsilon\Delta_A+o(\epsilon)\) and
\(B+\epsilon\Delta_B+o(\epsilon)\) of \(A\) and \(B\) on the eigenvalues
\(\lambda_s\) and their corresponding eigenvectors \(x_s\). Throughout
we assume that \(\lambda_s\) has multiplicity one, i.e.~that
\(\lambda_t\not=\lambda_s\) if \(t\not=s\). Note that this does not mean
all eigenvalues need to be different. Also note that if \(\epsilon\) is
small enough then \(B(\epsilon)\) is still positive definite and
\(\lambda_s(A(\epsilon),B(\epsilon))\) is still a simple eigenvalue.

To find \(d\lambda_s\) and \(dx_s\) we must solve the equations
\begin{subequations}
\begin{multline}
(A+\epsilon\Delta_A+o(\epsilon))(x_s+\epsilon dx_s+o(\epsilon))=\\(B+\epsilon\Delta_B+o(\epsilon)))(x_s+\epsilon dx_s+o(\epsilon))(\lambda_s+\epsilon d\lambda_s+o(\epsilon)),\label{eq-witheps1}
\end{multline}
and
\begin{equation}
(x_s+\epsilon dx_s+o(\epsilon))'(B+\epsilon\Delta_B+o(\epsilon))(x_s+\epsilon dx_s+o(\epsilon))=1.\label{eq-witheps2}
\end{equation}
\end{subequations}

Expand \eqref{eq-witheps1} and \eqref{eq-witheps2} and only keep the
first order terms. This gives \begin{subequations}
\begin{align}
Adx_s+\Delta_Ax_s&=d\lambda_s Bx_s+\lambda_sBdx_s+\lambda_s\Delta_Bx_s,\label{eq-fo1}\\
x_s'\Delta_Bx_s+2x_s'Bdx_s&=0.\label{eq-fo2}
\end{align}
\end{subequations} Premultiply equation \eqref{eq-fo1} with \(x_s'\).
This gives \begin{equation}
d\lambda_s=x_s'(\Delta_A-\lambda_s\Delta_B)x_s,\label{eq-def1}
\end{equation} We next solve for \(dx_s\). Write \(dx_s=X\alpha\), where
\(X\) is any complete set of eigenvectors from \(AX=BX\Lambda\),
normalized by \(X'BX=I\). Then \eqref{eq-fo1} becomes \begin{equation}
BX(\Lambda-\lambda_sI)\alpha=d\lambda_s Bx_s-(\Delta_Ax_s-\lambda_s\Delta_B)x_s.\label{eq-alp1}
\end{equation} Premultiplying by \(X'\) gives \begin{equation}
(\Lambda-\lambda_sI)\alpha=(d\lambda_s)e_s-X'(\Delta_A-\lambda_s\Delta_B)x_s,\label{eq-alp2}
\end{equation} with \(e_s\) a unit
vector.\footnote{A unit vector $e_s$ has element $s$ equal to one and all other elements equal to zero.}
Both sides of \eqref{eq-alp2} are vectors of length \(n\). Using
\eqref{eq-def1} we see that element \(s\) of both vectors is equal to
zero. For \(t\not= s\) we obtain \begin{equation}
\alpha_t=-\frac{x_t'(\Delta_A-\lambda_s\Delta_B)x_s}{\lambda_t-\lambda_s},\label{eq-alp3}
\end{equation} and, using \eqref{eq-fo2}, \begin{equation}
\alpha_s=-\frac12x_s'\Delta_Bx_s.\label{eq-alp3}
\end{equation} Thus \begin{equation}
dx_s=-\sum_{t\not= s}\frac{x_t'(\Delta_A-\lambda_s\Delta_B)x_s}{\lambda_t-\lambda_s}x_t-\frac12(x_s'\Delta_Bx_s)x_s.\label{eq-def2}
\end{equation} The two equations \eqref{eq-def1} and \eqref{eq-def2} are
the basic tools we use in this paper.

Equations \eqref{eq-def1} and \eqref{eq-def2} simplify in some important
special cases. For example, we can perturb \(A\) but not \(B\). Thus
\(\Delta_B=0\), and \begin{subequations}
\begin{equation}
d\lambda_s=x_s'\Delta_Ax_s\label{eq-sim1},
\end{equation}
and
\begin{equation}
dx_s=-\sum_{t\not= s}\frac{x_t'\Delta_Ax_s}{\lambda_t-\lambda_s}x_t.\label{eq-sim2}
\end{equation}
\end{subequations} If, in addition, \(B=I\) we have perturbation
equations for an SEV or simple eigenvalue problem. The case in which we
perturb \(B\) and not \(A\) is handled in the same way. Other special
cases and simplifications will be treated next.

\subsection{Parametric Perturbations}\label{sec-parametric}

If \(A\) and \(B\) are differentiable functions of a vector of \(q\)
parameters \(\theta\) then \begin{subequations}
\begin{align}
A(\theta+\epsilon d\theta)&=A(\theta)+\epsilon\sum_{r=1}^q d\theta_r\mathcal{D}_rA+o(\epsilon),\label{eq-par1}\\
B(\theta+\epsilon d\theta)&=B(\theta)+\epsilon\sum_{r=1}^q d\theta_r\mathcal{D}_rB+o(\epsilon).\label{eq-par2}
\end{align}
\end{subequations} Here the \(\mathcal{D}_rA\) is a matrix with partial
derivatives of \(A\) with respect to \(\theta_r\), evaluated at
\(\theta\), and \(d\theta_r\) is element \(r\) of the perturbation
\(d\theta\). And similarly for \(B\).

We now can apply equations \eqref{eq-def1} and \eqref{eq-def2} with
\begin{subequations}
\begin{align}
\Delta_A&=\sum_{r=1}^qd\theta_r\mathcal{D}_rA,\label{eq-par3}\\
\Delta_B&=\sum_{r=1}^qd\theta_r\mathcal{D}_rB.\label{eq-par4}
\end{align}
\end{subequations} If \(A\) and \(B\) depend on two different sets of
parameters then we can use the same equations with some of the
\(\mathcal{D}_rA\) and some of the \(\mathcal{D}_rB\) equal to zero.

\subsubsection{Linear Perturbations}\label{sec-perlinear}

In an important special case \(A\) and \(B\) are linear in \(\theta\).
So \begin{subequations}
\begin{align}
A(\theta)&=\sum_{r=1}^q\theta_r A_r,\label{eq-linpar1}\\
B(\theta)&=\sum_{r=1}^q\theta_r B_r.\label{eq-linpar2}
\end{align}
\end{subequations} In that case \(\Delta_A=\mathcal{D}_rA=A_r\) and
\(\Delta_B=\mathcal{D}_rB=B_r\).

\subsubsection{Elementwise Perturbations}\label{sec-perelementwise}

In an important special case of the linear case the parameters are all
the \(n(n+1)\) elements of \(A\) and \(B\) on and above the diagonal. We
have \begin{subequations}
\begin{align}
A&=\mathop{\sum\sum}_{1\leq i<j\leq n}a_{ij}E_{ij}+\sum_{i=1}^na_{ii}E_{i},\label{eq-elemper1}\\
B&=\mathop{\sum\sum}_{1\leq i<j\leq n}b_{ij}E_{ij}+\sum_{i=1}^nb_{ii}E_{i},\label{eq-elemper2}
\end{align}
\end{subequations} with \(\smash{E_{ij}:=e_ie_j'+e_je_i'}\) and
\(\smash{E_i:=e_ie_i'}\).

\subsection{Perturbation Code}\label{sec-pertcode}

The code in Section~\ref{sec-code} has the function perturbGeigen(),
written in R (R Core Team (\citeproc{ref-r_core_team_24}{2024})), which
has arguments \(a, b, da, db\) and \(p\). The first four arguments are
the values of \(A, B, \Delta_A,\Delta_B\). The remaining argument \(p\)
is a subset of \(\{1,2,\cdots,n\}\), with \(1\leq m\leq n\) elements.
The program computes \(d\lambda_s\) and \(dx_s\) for all \(s\in p\) and
returns them, respectively, in a vector with \(m\) elements and an
\(n\times m\) matrix. The computations use \eqref{eq-sim1} and
\eqref{eq-sim2}.

The code section also has the function perturbCheck(), with the same
arguments as perturbGeigen(), plus the additional argument \(eps\), the
value of \(\epsilon\). The function computes generalized eigenvalues and
eigenvectors of the pair \((A+\epsilon\Delta_A, B+\epsilon\Delta_B)\)
and compares them with the output of perturbGeigen().

Our example uses two matrices \(A\) and \(B\) of order three. They are

\begin{verbatim}
     [,1]      [,2]      [,3]     
[1,] +4.000000 +1.000000 +2.000000
[2,] +1.000000 +5.000000 +3.000000
[3,] +2.000000 +3.000000 +6.000000
\end{verbatim}

\begin{verbatim}
     [,1]      [,2]      [,3]     
[1,] +3.000000 -1.000000 -1.000000
[2,] -1.000000 +3.000000 -1.000000
[3,] -1.000000 -1.000000 +3.000000
\end{verbatim}

The generalized eigen decomposition of \(A\) and \(B\) has eigenvalues

\begin{verbatim}
[1] +9.080056 +0.859087 +0.560857
\end{verbatim}

and eigenvectors

\begin{verbatim}
     [,1]      [,2]      [,3]     
[1,] -0.539749 -0.410185 -0.201044
[2,] -0.575284 +0.290249 -0.291211
[3,] -0.608749 +0.015385 +0.359428
\end{verbatim}

The perturbations \(\Delta_A\) and \(\Delta_B\) we use are

\begin{verbatim}
     [,1]      [,2]      [,3]     
[1,] +1.000000 +0.000000 +0.000000
[2,] +0.000000 +0.000000 +0.000000
[3,] +0.000000 +0.000000 +0.000000
\end{verbatim}

\begin{verbatim}
     [,1]      [,2]      [,3]     
[1,] +0.000000 +0.000000 +0.000000
[2,] +0.000000 +0.000000 +0.000000
[3,] +0.000000 +0.000000 +0.000000
\end{verbatim}

So \(B\) is not perturbed at all. The output of perturbGeigen() is

\begin{verbatim}
$dl
[1] 0.29132948 0.16825178 0.04041874

$dx
             [,1]        [,2]         [,3]
[1,] -0.013607411 -0.04105592  0.120297564
[2,]  0.004107324 -0.06503134 -0.072930515
[3,]  0.004992561  0.11578144  0.003499643
\end{verbatim}

We run perturbCheck() with \(\epsilon\) equal to \(0.01\). The
generalized eigenvalues and eigenvectors of \(A+\epsilon\Delta_A\) and
\(B+\epsilon\Delta_B\) are

\begin{verbatim}
[1] +9.0829695442 +0.8607716139 +0.5612588419
\end{verbatim}

\begin{verbatim}
     [,1]          [,2]          [,3]         
[1,] -0.5398854997 -0.4105912934 -0.1998460332
[2,] -0.5752433377 +0.2896010798 -0.2919354673
[3,] -0.6086986633 +0.0165387642 +0.3594612728
\end{verbatim}

and the first order approximations \(\Lambda+\eps d\Lambda\) and
\(X+\eps dX\) are

\begin{verbatim}
[1] +9.0829688098 +0.8607699411 +0.5612612491
\end{verbatim}

\begin{verbatim}
     [,1]          [,2]          [,3]         
[1,] -0.5398855373 -0.4105956100 -0.1998411515
[2,] -0.5752433585 +0.2895982756 -0.2919398461
[3,] -0.6086986857 +0.0165432294 +0.3594628991
\end{verbatim}

The approximation is very good, although this is not surprising given
the small example and the separation of the eigenvalues. It does
indicate our formula are probably OK.

\sectionbreak

\section{Partial Derivatives}\label{sec-partial}

\subsection{Basic Partial Derivatives}\label{sec-parbasic}

The parametric perturbation results can be easily translated into the
language and notation for partial derivatives. Let's introduce the
notation first. If \(f\) is a function of a vector \(\theta\) then the
partial derivative with respect to \(\theta_r\) is \(\mathcal{D}_rf\),
defined by \[
\mathcal{D}_rf(\theta):=\lim_{\epsilon\rightarrow 0}\frac{f(\theta+\epsilon e_r)-f(\theta)}{\epsilon},
\] with \(e_r\) a unit vector.

To find partial derivatives we set \(d\theta=e_r\) In our perturbation
equations \eqref{eq-par3} and \eqref{eq-par4}. Thus we only perturb
\(\theta_r\). We then have \(\Delta_A=\mathcal{D}_rA\) and
\(\Delta_B=\mathcal{D}_rB\), and thus \begin{subequations}
\begin{equation}
\mathcal{D}_r\lambda_s=x_s'(\mathcal{D}_rA-\lambda_s\mathcal{D}_rB)x_s,\label{eq-parper1}
\end{equation}
and
\begin{equation}
\mathcal{D}_rx_s=-\sum_{t\not= s}\frac{x_t'(\mathcal{D}_rA-\lambda_s\mathcal{D}_rB)x_s}{\lambda_t-\lambda_s}x_t-\frac12(x_s'\mathcal{D}_rBx_s)x_s.\label{eq-parper2}
\end{equation}
\end{subequations} For linear parametric perturbations we have the same
equations for the partial derivatives with \(\mathcal{D}_rA=A_r\) and
\(\mathcal{D}_rB=B_r\).

\subsection{Elementwise Perturbations}\label{sec-parelementwise}

For elementwise perturbations there are some useful simplifications. If
we apply \eqref{eq-elemper1} and \eqref{eq-elemper2} to
\eqref{eq-parper1} and \eqref{eq-parper2} we get \begin{subequations}
\begin{equation}
\mathcal{D}_{ij}^A\lambda_s=\begin{cases}2x_{is}x_{js}&\text{ if }i\not= j,\\
x_{is}^2&\text{ if }i=j.
\end{cases}\label{eq-elempar1}
\end{equation}
\begin{equation}
\mathcal{D}_{ij}^B\lambda_s=\begin{cases}-2\lambda_sx_{is}x_{js}&\text{ if }i\not= j,\\
-\lambda_sx_{is}^2&\text{ if }i=j.
\end{cases}
\end{equation}\label{eq-elempar2}
\begin{equation}
\mathcal{D}_{ij}^Ax_s=
\begin{cases}
-\sum_{t\not= s}\frac{x_{is}x_{jt}+x_{js}x_{it}}{\lambda_t-\lambda_s}x_t&\text{ if }i\not= j,\\
-\sum_{t\not= s}\frac{x_{is}x_{it}}{\lambda_t-\lambda_s}x_t&\text{ if }i=j.
\end{cases}\label{eq-elempar3}
\end{equation}
\begin{equation}
\mathcal{D}_{ij}^Bx_s=
\begin{cases}
\lambda_s\sum_{t\not= s}\frac{x_{is}x_{jt}+x_{js}x_{it}}{\lambda_t-\lambda_s}x_t-x_{is}x_{js}x_s&\text{ if }i\not= j,\\
\lambda_s\sum_{t\not= s}\frac{x_{is}x_{it}}{\lambda_t-\lambda_s}x_t-\frac12x_{is}^2x_s&\text{ if }i=j.
\end{cases}\label{eq-elempar4}
\end{equation}
\end{subequations} In \eqref{eq-elempar1}-\eqref{eq-elempar4} we use the
somewhat ad-hoc notation \(\mathcal{D}_{ij}^A\) and
\(\mathcal{D}_{ij}^B\) for the partial derivatives with respect to
\(a_{ij}\) and \(b_{ij}\).

As an aside, instead of deriving \eqref{eq-elempar1}-\eqref{eq-elempar4}
from \eqref{eq-parper1} and \eqref{eq-parper2} we could also have used
the chain rule to derive \eqref{eq-parper1} and \eqref{eq-parper2} from
\eqref{eq-elempar1}-\eqref{eq-elempar4}. This looks like
\begin{subequations}
\begin{align}
\mathcal{D}_r\lambda_s&=\mathop{\sum\sum}_{1\leq i\leq j\leq n}\mathcal{D}_{ij}^A\lambda_s\mathcal{D}_ra_{ij}+\mathop{\sum\sum}_{1\leq i\leq j\leq n}\mathcal{D}_{ij}^B\lambda_s\mathcal{D}_rb_{ij},\label{eq-elemchain1}\\
\mathcal{D}_rx_s&=\mathop{\sum\sum}_{1\leq i\leq j\leq n}\mathcal{D}_{ij}^Ax_s\mathcal{D}_ra_{ij}+\mathop{\sum\sum}_{1\leq i\leq j\leq n}\mathcal{D}_{ij}^Bx_s\mathcal{D}_rb_{ij}.\label{eq-elemchain2}
\end{align}
\end{subequations}

\subsection{Second Order Partials}\label{sec-second}

For various purposes in data analysis, such as Newton's method or
asymptotic bias correction, we need the second derivatives of the
eigenvalues and eigenvectors.

We start by differentiating equation \eqref{eq-parper1} with respect to
\(\theta_u\). This gives

\begin{equation}
\mathcal{D}_{ru}\lambda_s=2(d_ux_s)'(\mathcal{D}_rA-\lambda_s\mathcal{D}_rB)x_s+
x_s'(\mathcal{D}_{ru}A-\lambda_s\mathcal{D}_{ru}B)x_s-d_u\lambda_sx_s'\mathcal{D}_rBx_s,
\label{eq-secparlbd}\end{equation}

We could expand this further by substituting \(d_t\lambda_s\) from
\(d_tx_s\) from \eqref{eq-parper1} and \eqref{eq-parper2}. But in
computation we will use \eqref{eq-secparlbd} as is, even though
\eqref{eq-secparlbd} does not show immediately that for each \(s\) the
Hessian \(\mathcal{D}_{ru}\lambda_s\) is a symmetric matrix of order
\(q\). Note that in the linear case
\(\mathcal{D}_{ru}A=\mathcal{D}_{ru}B=0\), so the middle term on the
right disappears.

The logical next step is to differentiate \eqref{eq-parper2} with
respect to \(\theta_u\). The resulting formula is pretty horrendous, but
think of it as a recipe for calculation, not as a beautiful object in
its own right. We compute the second partials for one single element
\(x_{ks}\) of \(X\) at a time. For each element \(x_{ks}\) the Hessian
will be a symmetric matrix of order \(q\). From \eqref{eq-parper2}
\begin{equation}
\mathcal{D}_rx_{ks}=-\sum_{t\not= s}\frac{x_t'(\mathcal{D}_rA-\lambda_s\mathcal{D}_rB)x_s}{\lambda_t-\lambda_s}x_{kt}-\frac12(x_s'\mathcal{D}_rBx_s)x_{ks}.\label{eq-parper3}
\end{equation}

We start by working on the first term on the right of
\eqref{eq-parper3}. Differentiating with respect to \(\theta_u\) gives
\begin{multline}
\mathcal{D}_{u}\left\{\frac{x_t'(\mathcal{D}_rA-\lambda_s\mathcal{D}_rB)x_s}{\lambda_t-\lambda_s}x_{kt}\right\}=\frac{x_t'(\mathcal{D}_rA-\lambda_s\mathcal{D}_rB)x_s}{\lambda_t-\lambda_s}d_ux_{kt}+
\\\mathcal{D}_{u}\left\{\frac{x_t'(\mathcal{D}_rA-\lambda_s\mathcal{D}_rB)x_s}{\lambda_t-\lambda_s}\right\}x_{kt}.\label{hess1}
\end{multline} The derivative in the second term on the right of
\eqref{hess1} evaluates to \begin{multline}
\mathcal{D}_{u}\left\{\frac{x_t'(\mathcal{D}_rA-\lambda_s\mathcal{D}_rB)x_s}{\lambda_t-\lambda_s}\right\}=\\
\frac{(\lambda_t-\lambda_s)\mathcal{D}_u\{x_t'(\mathcal{D}_rA-\lambda_s\mathcal{D}_rB)x_s\}-x_t'(\mathcal{D}_rA-\lambda_s\mathcal{D}_rB)x_s(d_u\lambda_t-d_u\lambda_s)}{(\lambda_t-\lambda_s)^2}\label{hess2}
\end{multline} The derivative in the first term of the numerator on the
right of \eqref{hess2} is \begin{multline}
\mathcal{D}_u\{x_t'(\mathcal{D}_rA-\lambda_s\mathcal{D}_rB)x_s\}=
(d_ux_t)'(\mathcal{D}_rA-\lambda_s\mathcal{D}_rB)x_s+\\
x_t'(\mathcal{D}_rA-\lambda_s\mathcal{D}_rB)d_ux_s+
x_t'(\mathcal{D}_{ru}A-\lambda_s\mathcal{D}_{ru}B)x_s-(d_u\lambda_s)x_t'(\mathcal{D}_rB)x_s.
\label{hess3}\end{multline} And finally, differentiating the last term
in \eqref{eq-parper3}, \begin{equation}
\mathcal{D}_u\{(x_s'D_rBx_s)x_{ks}\}=\{2(d_ux_s)'D_rBx_s+x_s'D_{ru}Bx_s\}x_{ks}+(x_s'D_rBx_s)d_ux_{ks}.
\end{equation} There are simplifications in the linear case, where the
second derivatives of \(A\) and \(B\) are zero, and we can replace
\(\mathcal{D}_rA\) and \(\mathcal{D}_rB\) by \(A_r\) and \(B_r\). And
there are more simplifications for SEV problems where \(B\) does not
depend on \(\theta\) and \(\mathcal{D}_rB=0\).

In the linear case, using the abbreviation \(C_{rs}:=A_r-\lambda_sB_r\)
we have \begin{equation}
\mathcal{D}_{ru}\lambda_s=2(d_ux_s)'C_{rs}x_s-d_u\lambda_sx_s'B_rx_s,
\end{equation} and \begin{align}
\mathcal{D}_{ru}x_{ks}&=-\sum_{t\not= s}\left\{\frac{x_t'C_{rs}x_s}{\lambda_t-\lambda_s}d_ux_{kt}\right.\notag\\
&+\frac{(d_ux_t)'C_{rs}x_s+
x_t'C_{rs}d_ux_s-(d_u\lambda_s)x_t'B_rx_s}{\lambda_t-\lambda_s}x_{kt}\notag\\
&\left.-\frac{x_t'C_{rs}x_s(d_u\lambda_t-d_u\lambda_s)}{(\lambda_t-\lambda_s)^2}x_{kt}\right\}\notag\\
&-((d_ux_s)'B_rx_s)x_{ks}-\frac12(x_s'B_rx_s)d_ux_{ks}.
\end{align}

\subsection{Partial Derivative Code}\label{sec-partialcode}

The functions partialGeigen() and partialCheck() can be used for linear
perturbations. They have both have arguments \(theta, a, b, s\), where
\(a\) and \(b\) are lists of matrices of lengths \(q\) qnd \(p\) and
\(theta\) is a vector of length \(p+q\). The last \(q\) matrices in the
list \(a\) are zero, as are the first \(p\) matrices in the list \(b\).
The index \(1\leq s\leq n\) dictates which eigen-pair we study.

partialGeigen() uses the formulas \eqref{eq-parper1} and
\eqref{eq-parper2}, while partialCheck computes numerical derivatives
using grad() and jacobian() from the numDeriv package (Gilbert and
Varadhan (\citeproc{ref-gilbert_varadhan_19}{2019})). For our example we
use the same \(A\) and \(B\) as before, and we use elementwise
perturbation. For the dominant eigenvalue partialGeigen() gives the
derivatives with respect to the elements of \(A\) and \(B\) as

\begin{verbatim}
     [,1]      [,2]      [,3]     
[1,] +0.291329 +0.621019 +0.657143
[2,] +0.621019 +0.330952 +0.700407
[3,] +0.657143 +0.700407 +0.370575
\end{verbatim}

\begin{verbatim}
     [,1]      [,2]      [,3]     
[1,] -2.645288 -5.638886 -5.966899
[2,] -5.638886 -3.005064 -6.359736
[3,] -5.966899 -6.359736 -3.364840
\end{verbatim}

We also used partialCheck() to compute numerical derivatives. The
maximum absolute difference between the numerical and analytical
partials of the eigenvalue is 0.0000000009.

The partialGeigen() function also gives the partials of the dominant
eigenvector. The partials of the three eigenvector elements with respect
to the elements of \(A\) are

\begin{verbatim}
     [,1]      [,2]      [,3]     
[1,] -0.013607 -0.010396 -0.010354
[2,] -0.010396 +0.004378 +0.009954
[3,] -0.010354 +0.009954 +0.005631
     [,1]      [,2]      [,3]     
[1,] +0.004107 -0.006526 +0.010971
[2,] -0.006526 -0.011622 -0.005542
[3,] +0.010971 -0.005542 +0.007149
     [,1]      [,2]      [,3]     
[1,] +0.004993 +0.011660 -0.002570
[2,] +0.011660 +0.006756 -0.001592
[3,] -0.002570 -0.001592 -0.009249
\end{verbatim}

and those with respect to \(B\) are

\begin{verbatim}
     [,1]      [,2]      [,3]     
[1,] +0.202179 +0.261993 +0.271365
[2,] +0.261993 +0.049566 +0.098643
[3,] +0.271365 +0.098643 +0.048881
     [,1]      [,2]      [,3]     
[1,] +0.046504 +0.237890 +0.089408
[2,] +0.237890 +0.200723 +0.251791
[3,] +0.089408 +0.251791 +0.041683
     [,1]      [,2]      [,3]     
[1,] +0.043340 +0.083153 +0.223351
[2,] +0.083153 +0.039392 +0.227640
[3,] +0.223351 +0.227640 +0.196773
\end{verbatim}

The maximum absolute differences between the numerical and analytical
partials of the eigenvector are, for \(A\) and \(B\),

\begin{verbatim}
0.000000000015786 0.000000000007729 0.000000000007234 
\end{verbatim}

\begin{verbatim}
0.000000000023673 0.000000000019420 0.000000000012131 
\end{verbatim}

In elementwise perturbation partialGeigen() uses equations
\eqref{eq-parper1} and \eqref{eq-parper2} and consequently needs lists
of binary sparse matrices as arguments. This is very wasteful, both in
memory and speed, given the fact that we also have the compact equations
\eqref{eq-elempar1} and \eqref{eq-elempar2}. We have added the more
specialized function partialElement() that uses these compact equations.
It does not use \(\theta\) and the lists with the \(A_r\) and \(B_r\).
It gives the same results as partialGeigen(), but is much faster.

The function hessianGeigenEval() computes the second partials of the
eigenvalues for linear perturbations. It has the same arguments as
partialGeigen(). The function returns a list with the second partials of
the eigenvalues.

For our small example

The second partials of the dominant eigenvalue with respect to the
elements of \(A\) and \(B\) are.

\begin{verbatim}
      [,1]     [,2]     [,3]     [,4]     [,5]     [,6]     [,7]     [,8]    
 [1,]   +0.019   +0.000   -0.005   +0.011   +0.008   -0.014   -0.558   -0.409
 [2,]   +0.000   +0.015   -0.010   +0.011   -0.013   +0.003   -0.409   -0.571
 [3,]   -0.005   -0.010   +0.009   -0.011   +0.006   +0.002   -0.385   -0.366
 [4,]   +0.011   +0.011   -0.011   +0.015   -0.005   -0.006   -0.283   -0.293
 [5,]   +0.008   -0.013   +0.006   -0.005   +0.013   -0.008   -0.274   -0.103
 [6,]   -0.014   +0.003   +0.002   -0.006   -0.008   +0.011   -0.101   -0.272
 [7,]   -0.558   -0.409   -0.385   -0.283   -0.274   -0.101   +8.571   +7.417
 [8,]   -0.409   -0.571   -0.366   -0.293   -0.103   -0.272   +7.417   +9.110
 [9,]   -0.385   -0.366   -0.568   -0.106   -0.290   -0.277   +7.448   +7.501
[10,]   -0.283   -0.293   -0.106   -0.218   -0.054   -0.053   +4.211   +4.398
[11,]   -0.274   -0.103   -0.290   -0.054   -0.231   -0.048   +4.351   +2.909
[12,]   -0.101   -0.272   -0.277   -0.053   -0.048   -0.240   +3.009   +4.680
      [,9]     [,10]    [,11]    [,12]   
 [1,]   -0.385   -0.283   -0.274   -0.101
 [2,]   -0.366   -0.293   -0.103   -0.272
 [3,]   -0.568   -0.106   -0.290   -0.277
 [4,]   -0.106   -0.218   -0.054   -0.053
 [5,]   -0.290   -0.054   -0.231   -0.048
 [6,]   -0.277   -0.053   -0.048   -0.240
 [7,]   +7.448   +4.211   +4.351   +3.009
 [8,]   +7.501   +4.398   +2.909   +4.680
 [9,]   +9.616   +2.820   +4.735   +4.873
[10,]   +2.820   +2.752   +1.361   +1.459
[11,]   +4.735   +1.361   +3.092   +1.549
[12,]   +4.873   +1.459   +1.549   +3.422
\end{verbatim}

There is also hessianCheckEval() that computes numerical second
partials. The maximum absolute difference between the numerical and
analytical second partials of the eigenvalue is 0.000000010647361.

\sectionbreak

\section{Generalized SVD}\label{sec-GSV}

Suppose \(F\) is an \(n\times m\) matrix, \(G\) is a positive definite
matrix of order \(n\), and \(H\) is a positive definite matrix of order
\(m\). We suppose without loss of generality that \(n\geq m\). The
generalized singular value problem for the triple \((F,G,H)\) is to find
solutions to the system \begin{subequations}
\begin{align}
Fy&=\lambda Gx,\label{eq-svd1}\\
F'x&=\lambda Hy,\label{eq-svd2}\\
x'Gx+y'Hy&=1,\label{eq-svd3}
\end{align}
\end{subequations} We refer to this as a GSV system, short for
generalized singular value system.

Now consider the GEV system \begin{subequations}
\begin{equation}
\begin{bmatrix}
0&F\\
F'&0
\end{bmatrix}
\begin{bmatrix}
x\\y
\end{bmatrix}
=\lambda
\begin{bmatrix}
G&0\\
0&H
\end{bmatrix}
\begin{bmatrix}
x\\y
\end{bmatrix}\label{eq-gsevd1},
\end{equation}
with normalization
\begin{equation}
x'Gx+y'Hy=1.
\end{equation}\label{eq-gsevd2}
\end{subequations} It is easy to see that \(x\) and \(y\) satisfy
\eqref{eq-gsevd1} and \eqref{eq-gsevd2} if and only if they satisfy
\eqref{eq-svd1} and \eqref{eq-svd2}.

In GSV system the normalization constraint \eqref{eq-svd3} is often
replaced by the constraint \(x'Gx=y'Hy=1\). We now show that this does
not change the solutions of the GSV system, except for multiplying the
singular vectors with a scale factor \(\frac12\sqrt{2}\).

The GSV system has \(r=\text{rank}(F)\) solutions with \(\lambda_s>0\).
For each of these solutions \((\lambda_s,x_s,y_s)\) there is a mirror
solution \((-\lambda_s,x_s,-y_s)\), and these \(2r\) solutions are also
solutions of the GEV system. Because in GEV two solutions with different
eigenvalues are orthogonal it follows that for a pair of mirror
solutions with non-zero eigenvalue \(x_s'Gx_s-y_s'Hy_s=0\), and thus,
using \eqref{eq-gsevd2}, \(x_s'Gx_s=y_s'Hy_s=\frac12\). In addition both
systems have \(n+m-2r\) solutions with zero eigenvalues, with
eigenvectors in the direct sum of the null spaces of \(F\) and \(F'\).

In summary, the solutions of the GEV and GSV systems for the ordered
eigenvalues are \begin{subequations}
\begin{equation}
\begin{bmatrix}
\Lambda&0&0&-\Lambda
\end{bmatrix},
\end{equation}
and for the corresponding eigenvectors
\begin{equation}
\begin{bmatrix}
X&X_\perp&0&\hfill X\\
Y&0&Y_\perp&-Y
\end{bmatrix},
\end{equation}
\end{subequations} with \(X_\perp\) a G-orthonomal basis for the
null-space of \(F'\) and \(Y_\perp\) an H-orthonormal basis for the
null-space of \(F\).

We now have enough information to apply our previous perturbation
results to GSV systems. We will only consider perturbations of the form
\[
\Delta_A=\begin{bmatrix}
0&\Delta_F\\
\Delta_F'&0
\end{bmatrix},
\] and \[
\Delta_B=\begin{bmatrix}
\Delta_G&0\\
0&\Delta_H
\end{bmatrix}
\] so that the perturbed system is still a GSV system.

From \eqref{eq-def1}we have \[
\delta\lambda_s=2x_s'\Delta_Fy_s-\lambda_s(x_s'\Delta_Gx_s+y_s'\Delta_Hy_s),
\]

Partial Derivatives

\section{Applications}\label{sec-applications}

\subsection{Principal Component Analysis}\label{sec-pca}

Suppose we have \(m\) numerical variables, with variable \(j\) having
\(k_j\) possible values. This defines \(q:=\smash{\prod_{j=1}^m k_j}\)
possible profiles, which are vectors of length \(m\) with all
combinations of the values. The data are the relative frequencies of the
profiles. Write \(G\) for the \(q\times m\) matrix of profiles.

The GEV problem for the principal component analysis (PCA) of a
covariance matrix has \[
A=G'(P-pp')G,\\
B=I
\] while for a PCA of the correlation matrix \[
A=G'(P-pp')G,\\
B=\text{diag}(G'(P-pp')G)
\] From \ldots{}\\
\[
\mathcal{D}_rA=g_rg_r'-(\mu e_r'+e_r\mu'),
\] with \(\mu:=Gp\).

From \ldots{} \(\mathcal{D}_rB=0\). From \ldots{} \[
\mathcal{D}_rB=\text{diag}(g_rg_r')-2\mu_r
\] These can be used in \ldots{} and \ldots{}

\subsection{Canonical Analysis}\label{sec-canonical}

In Canonical Analysis \[
A=\begin{bmatrix}
0&F'G\\
G'F&0
\end{bmatrix},
\] and \[
B=\begin{bmatrix}
F'F&0\\
0&G'G
\end{bmatrix}
\] Perturb \(F\) and \(G\), which gives \[
\Delta_A=\begin{bmatrix}
0&\Delta_F'G+F'\Delta_G\\
G'\Delta_F+F\Delta_G&0
\end{bmatrix}
\] and \[
\Delta_B=\begin{bmatrix}
\Delta_F'F+F'\Delta_F&0\\
0&\Delta_G'G+G'\Delta_G
\end{bmatrix}
\]

\subsection{Multiple Correspondence Analysis}\label{sec-mca}

We could introduce Multiple Correspondence Analysis (MCA) as a form of
canonical analysis and use the perturbation results from
Section~\ref{sec-canonical}. Instead we go directly to a parametric
approach.

Suppose we have \(m\) categorical variables, with variable \(j\) having
\(k_j\) categories. This defines \(q:=\smash{\prod_{j=1}^m k_j}\)
profiles, which are binary vectors of length
\(\smash{\sum_{j=1}^m k_j}\). The data are the relative frequencies of
the profiles (cf. Gifi (\citeproc{ref-gifi_B_90}{1990}), chapter 2).

In \[
A=\sum_{r=1}^qp_rg_rg_r',\\
B=\sum_{r=1}^qp_rG_r,
\] where \(p_r\) is the relative frequency of profile \(r\), and \(g_r\)
is the profile vector. Matri \(G_r\) is diagonal, with \(g_r\) on the
diagonal.

This is a linear parametric model, and consequently we can apply the
formulas from Section~\ref{sec-perlinear} to find the derivatives of the
eigenvalues and eigenvectors with respect to the \(p_r\).

\begin{subequations}
\begin{equation}
\mathcal{D}_r\lambda_s=x_s'(g_rg_r'-\lambda_s\text{diag}(g_rg_r'))x_s,\label{eq-linper1}
\end{equation}
and
\begin{equation}
\mathcal{D}_rx_s=-\sum_{t\not= s}\frac{x_t'(A_r-\lambda_sB_r)x_s}{\lambda_t-\lambda_s}x_t-\frac12(x_s'B_rx_s)x_s.\label{eq-linper2}
\end{equation}
\end{subequations}

\subsection{Classical Multidimensional
Scaling}\label{classical-multidimensional-scaling}

In classical multidimensional scaling (MDS) we have a symmetric matrix
\(D\) of squared dissimilarities.

\[
\xi_{ij}=-\frac12\left\{\theta_{ij}^2-\frac{1}{n}\sum_{l=1}^n\theta_{il}^2-\frac{1}{n}\sum_{l=1}^n\theta_{lj}^2
+\frac{1}{n^2}\sum_{k=1}^n\sum_{l=1}^n\theta_{kl}^2\right\}
\] \[
A=\mathop{\sum\sum}_{1\leq i<j\leq n}\xi_{ij}E_{ij}+\sum_{i=1}^n\xi_{ii}E_i
\]

\section{Factor Analysis}\label{sec-fa}

\section{Low rank Matrix Approximation}\label{sec-rank}

\section{Discussion}\label{sec-discussion}

If \(B\) is a singular, if \(B\) is indefinite. If \(A\) and \(B\) are
not symmetric.

\section{Code}\label{sec-code}

\section*{References}\label{sec-references}
\addcontentsline{toc}{section}{References}

\phantomsection\label{refs}
\begin{CSLReferences}{1}{0}
\bibitem[\citeproctext]{ref-deleeuw_R_07c}
De Leeuw, J. 2007. {``Derivatives of Generalized Eigen Systems with
Applications.''} Preprint Series 528. Los Angeles, CA: UCLA Department
of Statistics.
\url{https://jansweb.netlify.app/publication/deleeuw-r-07-c/deleeuw-r-07-c.pdf}.

\bibitem[\citeproctext]{ref-gifi_B_90}
Gifi, A. 1990. \emph{Nonlinear Multivariate Analysis}. New York, N.Y.:
Wiley.

\bibitem[\citeproctext]{ref-gilbert_varadhan_19}
Gilbert, P., and R. Varadhan. 2019. \emph{{numDeriv: Accurate Numerical
Derivatives}}. \url{https://CRAN.R-project.org/package=numDeriv}.

\bibitem[\citeproctext]{ref-kato_76}
Kato, T. 1976. \emph{Perturbation Theory for Linear Operators}. Second
Edition. Springer.

\bibitem[\citeproctext]{ref-r_core_team_24}
R Core Team. 2024. \emph{R: A Language and Environment for Statistical
Computing}. {Vienna, Austria}: R Foundation for Statistical Computing.
\url{https://www.R-project.org/}.

\bibitem[\citeproctext]{ref-wilkinson_65}
Wilkinson, J. H. 1965. \emph{{The Algebraic Eigenvalue Problem}}.
Clarendon Press.

\end{CSLReferences}




\end{document}
